\documentclass[a4paper]{ctexart}
\usepackage{indentfirst}
\usepackage{geometry}
\geometry{a4paper,scale=0.9}
\setlength{\parindent}{2em}
\author{李帅}
\title{逻辑分析仪使用总结}
\begin{document}  
\maketitle
\section{Logic Analyzer}
\subsection{基本设置}
\indent 在桌面上找到逻辑分析仪软件图标,双击打开后主界面如图。\\
\center
\includegraphics[width=4in,height=3in]{pictures-5.eps}\\
\indent 在采样ADC数字码之前需要先设置各位数字码对应的数据通道,单击\includegraphics[width=0.25in,heigh=0.25in]{inco.eps}
打开设置界面如图
\center
\includegraphics[width=4in,height=3in]{pictures-0.eps}\\
在设置界面由两个选项卡,第一个选项卡设置每一位数字码对应的数据通道。如图所示,第一列为总线/信号名称,右击通道名可添加或删除总线/信号。第二列是每个通道的标号,图中“POD A1[7:0]”即对应图中“Slot A Pod 1”中有$\surd$的通道,单击相应通道可以选择或取消相应通道,根据实际情况选择通道。一个设置示例如下图:
\includegraphics[width=4in,height=3in]{pictures-1.eps}\\
其中数字序号与$\surd$一样也是选中通道,只是同时也为通道排序,对于ADC的测试来说,每一位对应的权重是不同的,因而此处应该用这种方式选择通道,0代表
最高位,其他位按序号权重递减,由于后面采样到数据后还需要用Matlab处理,为了符合处理程序读取数据的格式,此处为每一位数据均分配了一个信号位,每个信
号对应的通道用$\surd$选中,对于信号位来说只有一位,因而选择一位通道即可。以上的设置中一定要注意通道排序与实际ADC输出位序的对应关系。逻辑分析仪可
以观察采样数据的二进制权重相加后的波形图,第一行设置为总线信号即是为了观察波形图,另外还要注意的是,如果要观察输出时钟信号,则可以添加一个信号位
并分配相应的时钟通道。这里要注意时钟通道与数据通道的对应关系。图示里由于使用POD 3采样信号及时钟,因而时钟选择A3,数据位均为POD 3上的通道位。\\
\indent 下面选择sampling选项卡设置采样选项。\\
\center
\includegraphics[width=4in,height=3in]{pictures-13.eps}\\
Acquisition里面可以选择同步采样和异步采样,Timing是异步采样,State是同步采样。如果要看输出时钟应选择异步采样,否则一般选择同步采样,另外对于芯片
没有输出时钟则必须选择异步采样。对于同步采样应在最下面的方框内选择时钟及触发沿。对于异步采样,应在Timing Options里面选择逻辑分析仪采样输出信号的频
率。\leftline{Acquisition Depth是采样数据的长度。}
\end{document}
